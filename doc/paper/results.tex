\section{Results}
\label{sec:results}

In the course of this project, it turned out that at least two of the three
approaches we initially identified are currently unfeasible. The idiomatic
solution using Flink's delta iterations can not be fully implemented, because
nested iterations, which are required to orthogonalize the Lanczos vectors and
thus ensure the correctness of the algorithm's results, are not supported yet
by Flink as of version 0.8.

As described in section \ref{ssec:loop_performance_problem}, the loop-based
approach we implemented is not feasible in practice due to major performance
issues. Our explanation for the observed problems is that the Flink compiler
and/or optimizer is overloaded by the complexity of the execution graph that it
has to generate or optimize, respectively. The problem here is most likely that
Flink can't really recognize the iterative nature of the algorithm. Instead,
before trying to build the execution graph, Flink probably already executes all
loop iterations in oder to determine, which and how many data sets are needed
and how the data flows through the program. At one point during the
investigation of this problem, we looked at a JSON representation of the
program's execution graph. Even though we only used three Lanczos iterations,
the representation of the graph consisted of \textit{several thousand} lines
of JSON. While one single iteration step in our implementation may not seem to
be that complex, we assume that Flink can't handle the complexity of the full
data flow that only unfolds during the attempt of compiling the execution
graph. This explanation seems even more probable considering that the
orthogonalization in every single iteration step is done in a nested loop that
iterates over all previously calculated results again, which adds a lot of
additional complexity to every single iteration.

We had initially discarded the idea of simply using Mahout's implementation of
the Lanczos algorithm by implementing all the interfaces it uses (see section
\ref{ssec:approach_mahout}). But as it stands at the end of our project, this
is the only one of the three identified approaches that could work and be
feasible in current versions of Flink. Unfortunately, we didn't have enough
time left to implement this approach as well and to actually test how well it's
working.
